%%%%%%%%%%%%%%%%%%%%%%%%%%%%%%%%%%%%%%%%%
% Lachaise Assignment
% LaTeX Template
% Version 1.0 (26/6/2018)
%
% This template originates from:
% http://www.LaTeXTemplates.com
%
% Authors:
% Marion Lachaise & François Févotte
% Vel (vel@LaTeXTemplates.com)
%
% License:
% CC BY-NC-SA 3.0 (http://creativecommons.org/licenses/by-nc-sa/3.0/)
% 
%%%%%%%%%%%%%%%%%%%%%%%%%%%%%%%%%%%%%%%%%

%----------------------------------------------------------------------------------------
%	PACKAGES AND OTHER DOCUMENT CONFIGURATIONS
%----------------------------------------------------------------------------------------

\documentclass{article}

%%%%%%%%%%%%%%%%%%%%%%%%%%%%%%%%%%%%%%%%%
% Lachaise Assignment
% Structure Specification File
% Version 1.0 (26/6/2018)
%
% This template originates from:
% http://www.LaTeXTemplates.com
%
% Authors:
% Marion Lachaise & François Févotte
% Vel (vel@LaTeXTemplates.com)
%
% License:
% CC BY-NC-SA 3.0 (http://creativecommons.org/licenses/by-nc-sa/3.0/)
% 
%%%%%%%%%%%%%%%%%%%%%%%%%%%%%%%%%%%%%%%%%

%----------------------------------------------------------------------------------------
%	PACKAGES AND OTHER DOCUMENT CONFIGURATIONS
%----------------------------------------------------------------------------------------

\usepackage{amsmath,amsfonts,stmaryrd,amssymb} % Math packages

\usepackage{enumerate} % Custom item numbers for enumerations

\usepackage[ruled]{algorithm2e} % Algorithms

\usepackage[framemethod=tikz]{mdframed} % Allows defining custom boxed/framed environments

\usepackage{listings} % File listings, with syntax highlighting
\lstset{
	basicstyle=\ttfamily, % Typeset listings in monospace font
}

\usepackage{derivative} % support for derivatives
\usepackage{siunitx} % support for SI units

\usepackage{empheq} 
\usepackage{xcolor}
\definecolor{lightgreen}{HTML}{90EE90}

%----------------------------------------------------------------------------------------
%	DOCUMENT MARGINS
%----------------------------------------------------------------------------------------

\usepackage{geometry} % Required for adjusting page dimensions and margins

\geometry{
	paper=a4paper, % Paper size, change to letterpaper for US letter size
	top=2.5cm, % Top margin
	bottom=3cm, % Bottom margin
	left=2.5cm, % Left margin
	right=2.5cm, % Right margin
	headheight=14pt, % Header height
	footskip=1.5cm, % Space from the bottom margin to the baseline of the footer
	headsep=1.2cm, % Space from the top margin to the baseline of the header
	%showframe, % Uncomment to show how the type block is set on the page
}

%----------------------------------------------------------------------------------------
%	FONTS
%----------------------------------------------------------------------------------------

\usepackage[utf8]{inputenc} % Required for inputting international characters
\usepackage[T1]{fontenc} % Output font encoding for international characters

\usepackage{XCharter} % Use the XCharter fonts

%----------------------------------------------------------------------------------------
%	COMMAND LINE ENVIRONMENT
%----------------------------------------------------------------------------------------

% Usage:
% \begin{commandline}
%	\begin{verbatim}
%		$ ls
%		
%		Applications	Desktop	...
%	\end{verbatim}
% \end{commandline}

\mdfdefinestyle{commandline}{
	leftmargin=10pt,
	rightmargin=10pt,
	innerleftmargin=15pt,
	middlelinecolor=black!50!white,
	middlelinewidth=2pt,
	frametitlerule=false,
	backgroundcolor=black!5!white,
	frametitle={Command Line},
	frametitlefont={\normalfont\sffamily\color{white}\hspace{-1em}},
	frametitlebackgroundcolor=black!50!white,
	nobreak,
}

% Define a custom environment for command-line snapshots
\newenvironment{commandline}{
	\medskip
	\begin{mdframed}[style=commandline]
}{
	\end{mdframed}
	\medskip
}

%----------------------------------------------------------------------------------------
%	FILE CONTENTS ENVIRONMENT
%----------------------------------------------------------------------------------------

% Usage:
% \begin{file}[optional filename, defaults to "File"]
%	File contents, for example, with a listings environment
% \end{file}

\mdfdefinestyle{file}{
	innertopmargin=1.6\baselineskip,
	innerbottommargin=0.8\baselineskip,
	topline=false, bottomline=false,
	leftline=false, rightline=false,
	leftmargin=2cm,
	rightmargin=2cm,
	singleextra={%
		\draw[fill=black!10!white](P)++(0,-1.2em)rectangle(P-|O);
		\node[anchor=north west]
		at(P-|O){\ttfamily\mdfilename};
		%
		\def\l{3em}
		\draw(O-|P)++(-\l,0)--++(\l,\l)--(P)--(P-|O)--(O)--cycle;
		\draw(O-|P)++(-\l,0)--++(0,\l)--++(\l,0);
	},
	nobreak,
}

% Define a custom environment for file contents
\newenvironment{file}[1][File]{ % Set the default filename to "File"
	\medskip
	\newcommand{\mdfilename}{#1}
	\begin{mdframed}[style=file]
}{
	\end{mdframed}
	\medskip
}

%----------------------------------------------------------------------------------------
%	NUMBERED QUESTIONS ENVIRONMENT
%----------------------------------------------------------------------------------------

% Usage:
% \begin{question}[optional title]
%	Question contents
% \end{question}

\mdfdefinestyle{question}{
	innertopmargin=1.2\baselineskip,
	innerbottommargin=0.8\baselineskip,
	roundcorner=5pt,
	nobreak,
	singleextra={%
		\draw(P-|O)node[xshift=1em,anchor=west,fill=white,draw,rounded corners=5pt]{%
		Question \theQuestion\questionTitle};
	},
}

\newcounter{Question} % Stores the current question number that gets iterated with each new question

% Define a custom environment for numbered questions
\newenvironment{question}[1][\unskip]{
	\bigskip
	\stepcounter{Question}
	\newcommand{\questionTitle}{~#1}
	\begin{mdframed}[style=question]
}{
	\end{mdframed}
	\medskip
}

%----------------------------------------------------------------------------------------
%	WARNING TEXT ENVIRONMENT
%----------------------------------------------------------------------------------------

% Usage:
% \begin{warn}[optional title, defaults to "Warning:"]
%	Contents
% \end{warn}

\mdfdefinestyle{warning}{
	topline=false, bottomline=false,
	leftline=false, rightline=false,
	nobreak,
	singleextra={%
		\draw(P-|O)++(-0.5em,0)node(tmp1){};
		\draw(P-|O)++(0.5em,0)node(tmp2){};
		\fill[black,rotate around={45:(P-|O)}](tmp1)rectangle(tmp2);
		\node at(P-|O){\color{white}\scriptsize\bf !};
		\draw[very thick](P-|O)++(0,-1em)--(O);%--(O-|P);
	}
}

% Define a custom environment for warning text
\newenvironment{warn}[1][Warning:]{ % Set the default warning to "Warning:"
	\medskip
	\begin{mdframed}[style=warning]
		\noindent{\textbf{#1}}
}{
	\end{mdframed}
}

%----------------------------------------------------------------------------------------
%	INFORMATION ENVIRONMENT
%----------------------------------------------------------------------------------------

% Usage:
% \begin{info}[optional title, defaults to "Info:"]
% 	contents
% 	\end{info}

\mdfdefinestyle{info}{%
	topline=false, bottomline=false,
	leftline=false, rightline=false,
	nobreak,
	singleextra={%
		\fill[black](P-|O)circle[radius=0.4em];
		\node at(P-|O){\color{white}\scriptsize\bf i};
		\draw[very thick](P-|O)++(0,-0.8em)--(O);%--(O-|P);
	}
}

% Define a custom environment for information
\newenvironment{info}[1][Info:]{ % Set the default title to "Info:"
	\medskip
	\begin{mdframed}[style=info]
		\noindent{\textbf{#1}}
}{
	\end{mdframed}
}

%----------------------------------------------------------------------------------------
%	EQUATION BOXES
%----------------------------------------------------------------------------------------
 % Include the file specifying the document structure and custom commands

%----------------------------------------------------------------------------------------
%	ASSIGNMENT INFORMATION
%----------------------------------------------------------------------------------------

\title{Critical Systems Evaluation} % Title of the assignment

\author{First Homework List\\ \texttt{acn2}} % Author name and email address

\date{CIN, UFPE --- \today} % University, school and/or department name(s) and a date

%----------------------------------------------------------------------------------------

\begin{document}

\maketitle % Print the title


\setcounter{Question}{0}
\begin{question}
    Explain the concepts of 
    \begin{enumerate}[{(a)}]
        \item fault, error, and failure
        \item reliability, steady-state availability and instantaneous availability
        \item  hazard and cumulative hazard functions
    \end{enumerate}
    \end{question}
\setcounter{Question}{1}

\begin{enumerate}[{(a)}]
    \item \textbf{fault} is the adjudged or hypothesized cause of an error, \textbf{error} is a state of a component of a system (a system substate) that may cause a subsequent failure and a \textbf{failure} is what occurs when an error reaches the system interface and alters the service.
    \item \textbf{reliability} is the probability that the system S does not fail up to time t, \textbf{steady-state availability} is something that is possible to quantify when system approaches stationary states, the formula for it is $A=\lim_{t \to \infty} A(t), t\ge0$ and \textbf{instantaneous availability} is the probability that the system is operational at time t.
    \item \textbf{hazard function} is the probability of the system S failing during the interval $[t,t+\delta t]$ if it has survived to the time t and \textbf{cumulative hazard function} is the total hazard until the given point in time.
\end{enumerate}
    
\begin{question}
Consider a database system with time to failure distribution represented
by the $CDF$ $F(t)=1+e^{-0.0005t}-2e^{-0.00025t}$.  
    \begin{enumerate}[{(a)}]
        \item Calculate the reliability at $t=5,500h$ 
        \item and compute the $MTTF$,
        \item obtain the system hazard function $\lambda(t)$  After, calculate $\lambda(t)$ at $t=5,500h$
    \end{enumerate}
\end{question}

\begin{enumerate}[{(a)}]
    \item The reliability of a system can be described by the following equation:
    \begin{equation}
        R(t) = 1 - F(t)
    \end{equation}
    We know that $F(t)=1+e^{-0.0005t}-2e^{-0.00025t}$, thus
    \begin{align*}
         R(t)&=1-(1+e^{-0.0005t}-2e^{-0.00025t}) \\
         &=1-1-e^{-0.0005t}+2e^{-0.00025t}  \\
        &=-e^{-0.0005t}+2e^{-0.00025t} \\
        &=2e^{-0.00025t}-e^{-0.0005t}
    \end{align*}
    For $t=5,500h$  we have
   \begin{empheq}[box=\fbox]{align*}
        R(t=5500)=2e^{-0.00025\cdot5500}-e^{-0.0005\cdot5500} \approx 0,442
    \end{empheq}
    
    \item The Mean Time to Failure ($MTTF$) can be described by the following equation
    \begin{equation}
        \int_{0}^{\infty} R(t) dt
    \end{equation}
    We know that $R(t)=2e^{-0.00025t}-e^{-0.0005t}$, thus
    \begin{empheq}[box=\fbox]{align*}
        \int_{0}^{\infty} R(t) dt &= \int_{0}^{\infty}2e^{-0.00025t}-e^{-0.0005t}dt \\
        &= \int_{0}^{\infty}2e^{-0.00025t}dt - \int_{0}^{\infty}e^{-0.0005t}dt \\
        &= 8000 - 4000 \\
        &= 4000h
    \end{empheq}

    \item We know that $\lambda(t)$ can be described as
    \begin{equation}
        \lambda(t) = -frac{dR(t)}{dt} \cdot \frac{1}{R(t)}
    \end{equation}
    and we know that $R(t)=2e^{-0.00025t}-e^{-0.0005t}$, thus
    \begin{align*}
        \lambda(t) &= -\frac{dR(t)}{dt} \cdot \frac{1}{R(t)} \\
        &=- \frac{\frac{d}{dt}(2e^{-0.00025t}-e^{-0.0005t})}{{2e^{-0.00025t}-e^{-0.0005t}}}\\
    \end{align*}
    so
    \begin{empheq}[box=\fbox]{align*}
        \lambda(t)=-\frac{0.0005e^{-0.0005t}-0.0005e^{-0.00025t}}{2e^{-0.00025t}-e^{-0.0005t}}
    \end{empheq}
    Substituting $t=5500$ on the obtained system hazard funciton we have
    \begin{empheq}[box=\fbox]{align*}
        \lambda(t=5500) &= -\frac{0.0005e^{-0.0005\cdot5500}-0.0005e^{-0.00025\cdot5500}}{2e^{-0.00025\cdot5500}-e^{-0.0005\cdot5500}} \\
        &= 0.0002138 \\
        &= 2.138 \cdot 10^{-4}/h
    \end{empheq}
\end{enumerate}

\setcounter{Question}{2}
\begin{question}
Assume the system reliability in $t = 720h$ is $7,500 DPM$. Calculate the probability of failure by $t=720 h$
\end{question}

The reliability of a system can be described by the following equation
    \begin{equation}
        R(t) = 1 - DPM\cdot10^{-6}
    \end{equation}
And the probability of failure of a system can be described as
    \begin{equation}
        F(t) = 1 - R(t)
    \end{equation}
And the equation to convert $R(t)$ to $DPM$ is
    \begin{equation}
        R(t) = 1 - DPM\cdot10^{-6}
    \end{equation}
For $t=720h$ and DPM = $7500DPM$ we have
    \begin{empheq}[box=\fbox]{align*}
        F(t=720)&=1-(1-R(t=720)) \\
        &= 1 - 1+(7500\cdot10^{-6})\\
        &= 7.5\cdot10^{-3}
    \end{empheq}

\setcounter{Question}{3}
\begin{question}
    Assume the reliability of a system that is represented by
    \[
    R(t) =
    \left\{ 
        \begin{array}{1}
            \frac{1-0.0004t}{0.96}\quad100\leq t\leq2500 &\\
            0\quad\quad t>2500
        \end{array} 
    \right.
    \]
    Obtain the density function of the time to failure and calculate the $MTTF$ and $MedTTF$.
\end{question}

The probability of failure of a system cam be described by the following equation
    \begin{equation}
        F(t)=1-R(t)
    \end{equation}
Therefore,
    \[
    F(t) =
    \left\{ 
        \begin{array}{1}
            1-\frac{1-0.0004t}{0.96}\quad100\leq t\leq2500 &\\
            1\quad\quad t>2500
        \end{array} 
    \right.
    \]
The density function of the time to failure can be described by the following equation
    \begin{equation}
        f(t)=\frac{d}{dt}F(t)
    \end{equation}
Thus,
    \[
    f(t) =
    \left\{ 
        \begin{array}{1}
            0.0004\quad100\leq t\leq2500 &\\
            0\quad\quad t>2500
        \end{array} 
    \right.
    \]
The mean time to fail ($MTTF$) can be described as
    \begin{equation}
        MTTF=\int_{0}^{\infty}tf(t)dt
    \end{equation}   
Therefore,
    \begin{empheq}[box=\fbox]{align*}
        MTTF&=\int_{100}^{2500}0.004t dt+ \int_{2500}^{\infty} 0\cdot t dt\\
        &=12500+0 \\
        &=12500h
    \end{empheq}
The median time to failure ( $MedTTF$ ) is  defined by the time instant, t, when $R(t)=0.5$, therefore

    \begin{align*}
        &R(t)=0.5=\frac{1=0.0004t}{0.96} \\
        & 1-0.004t = 0.96\cdot0.50 = 0.48 \\
        & -0.004t = 0.48 - 1 \\
        & -0.004t = -0.052 \\
        & t = \frac{0.052}{0.004} \\
        & t = 130h 
    \end{align*}

So the value for $MedTTF$ is
    \begin{empheq}[box=\fbox]{align*}
        MedTTF = 130h
    \end{empheq}

\end{document}

